\documentclass[a4paper, 12pt]{extarticle}

\usepackage[utf8]{inputenc}
\usepackage[russian]{babel}

\usepackage{hyperref}
\usepackage{multirow} 
\usepackage{graphicx}
\usepackage{bm}
\usepackage{geometry}

\geometry{a4paper,top=1.5cm,bottom=1.5cm,bindingoffset=0cm}
\geometry{left=2cm,textwidth=18cm}
% \linespread{1.0}
  
\usepackage{verbatim}

\title{}
\author{}
\date{}

\begin{document}

\section*{Оформление и сдача домашних заданий}

\subsection*{Задания, помеченные mail}

Такие задания отправляются на почту \texttt{filatovaur@gmail.com}. Необходимо указать тему письма \texttt{<Группа>, <Фамилия>, <Задача>}.
Например, правильно будет указать \texttt{11111, Филатов, 1.2}. Если в течение пары дней вы не получаете ответное письмо, то напомните о нем через другой канал связи (на лекции или семинаре, через группу в telegram), письмо могло быть ошибочно определено как вредоносное. Такое периодически случается даже с университетскими аккаунтами.

\subsection*{Задания, помеченные code}

По ссылке из слайдов вы найдете описание задания и процедуру сдачи. Простые задания обычно принимаются при личном общении на семинаре, более сложные могут потребовать ревью в GitHub. Уточняйте у проверяющего вашу задачу. Успешная или частичная сдача таких задач добавит вам баллов за курс, что влияет на итоговую оценку.

\subsection*{Задания, помеченные critical}

По ссылке из слайдов вы найдете описание задания и процедуру сдачи. Обратите внимание, что critical задачи являются предусловием для досрочного экзамена, поэтому у них есть т.н. soft deadline. У таких задач есть уровни сложности (base, medium, advanced), мы говорим, что ''задача сдана'', если зачтен хотя бы уровень base.  У каждой critical задачи есть hard deadline, если вы к его наступлению задачу не сдали, то далее она не принимается. Если сдали, но хотите повысить балл, то досдача после крайнего срока обговаривается индивидуально с проверяющим/лектором. Если у слушателя курса не сдана critical задача, то итоговая оценка сильно пострадает.

\subsection*{Система оценивания}

Запутанная. Есть два сценария:
\begin{itemize} 
 \item \textit{обычный}, когда слушатель курса ходит на лекции, выполняет задания и, в зависимости от количества сделанных задач и качества ответа на устном экзамене, получает оценку.
 \item \textit{приоритетный}, когда слушатель прикладывает чуть больше усилий в течение семестра (укладывается в soft deadline, посещает досрочный экзамен) и тогда уже в середине семестра у него появляется т.н. ''несгораемая оценка''. Иными словами, можно уже к середине семестра наработать на ''тройку'' и больше не появляться на занятиях, игнорировать critical задачи, оценка всё равно будет выставлена.
\end{itemize}

Как это работает:

\begin{itemize}

\item В первом блоке лекций выдается одна critical задача. Все, кто успеет её сдать до soft deadline, приглашаются на досрочный экзамен по первому блоку. Экзаменуемый, успешно ответивший на билет, получает несгораемую ''удовлетворительно''.

\item Во втором блоке лекций выдается одна critical задача. Все, кто преуспел на предыдущем шаге и успел сдать вторую задачу до soft deadline, приглашаются на досрочный экзамен по второму блоку. Экзаменуемый, успешно ответивший на билет, получает несгораемую ''хорошо''.

\item В третьем блоке лекций выдается одна critical задача. Все, кто преуспел на предыдущем шаге и успел сдать третью задачу до soft deadline, приглашаются на досрочный экзамен по третьему блоку. Экзаменуемый, успешно ответивший на билет, получает несгораемую ''отлично''.

\end{itemize}

Слушатели, не имеющие несгораемых оценок, или желающие улучшить оценку, следуют общему алгоритму:

\begin{itemize}	
	\item За решение задач в течение семестра начисляются баллы.
	\item За активное участие в лекциях начисляются очки.
	\item В конце семестра происходит устный экзамен по всем темам.
	\item Каждый слушатель получает две оценки: ''за практику'' (определяется набранными баллами) и ''за теорию'' (определяется ответом и набранными очками).
	\item Если есть хотя бы один ''неуд'', то он и выставляется в ведомость. 
	\item Если у слушателя курса не сдана хотя бы одна critical задача, то больше оценки ''удовлетворительно'' за курс не выставляется.
	\item Иначе выставляется среднее из двух оценок, с округлением вверх (\texttt{3+5=4}, \texttt{4+5=5}).
\end{itemize}


\end{document}
